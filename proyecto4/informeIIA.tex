\documentclass[a4paper, 8pt]{article}
\usepackage[T1]{fontenc}
\usepackage[utf8]{inputenc}
\usepackage{lmodern}
\usepackage[spanish]{babel}
\usepackage{enumerate}

	\addtolength{\oddsidemargin}{-.875in}
	\addtolength{\evensidemargin}{-.875in}
	\addtolength{\textwidth}{1.75in}

	\addtolength{\topmargin}{-.875in}
	\addtolength{\textheight}{1.75in}

\title{Trabajo práctico de lógica borrosa \\ Introducción a la inteligencia artificial}

\date{}

\begin{document}

\maketitle
\section{Explicación del problema}

	El objetivo de nuestro análisis consiste en estimar el grado de saturación por día, que sufre una persona luego de finalizar la lectura de un cuento o novela. Para ello, tenemos en cuenta los siguientes factores:
\begin{enumerate}[a)]
  \item El tiempo libre por día que tiene el lector.
  \item El plazo máximo para finalizar la lectura.
  \item El tiempo esperado para finalizar la lectura.
\end{enumerate}

Previamente, obtenemos el pto. (c), mediante la información dada por:
\begin{enumerate}[a)]
\setcounter{enumi}{3}
  \item El número de páginas del libro.
  \item La edad del lector.
  \item El grado de interés del lector en el libro.
\end{enumerate}

\section{Etapas del análisis}

\subsection{Primera etapa}

Utilizamos las siguientes variables lingüísticas:

\begin{itemize}
\item Las páginas (T)

  Rango de T: [0..1000] hojas.

  Conjuntos borrosos:
  
  \begin{itemize}
        \item \textit{Pequeño} 0 - 450
        \item \textit{Medio} 250 - 700
        \item \textit{Grande} 500 - 1000
  \end{itemize}
  
\item La edad (E). 

  Rango de E: [10..90] años.
  
  Conjuntos borrosos:
  
  \begin{itemize}
        \item \textit{Joven} 10 - 30
        \item \textit{Joven\_Adulto} 20 - 50
        \item \textit{Adulto} 35 - 90
  \end{itemize}
        
\item El interés (I).

  Rango de I: [0..100] por ciento.
  
  Conjuntos borrosos:
  
  \begin{itemize}
        \item \textit{Bajo} 0 - 40
        \item \textit{Medio} 30 - 70
        \item \textit{Alto} 60 - 100
  \end{itemize}
  
\item El tiempo de lectura (TF).

  Rango de TF: [0..200] horas.

  Conjuntos borrosos:
  
  \begin{itemize}
        \item \textit{Bajo} 0 - 30
        \item \textit{Medio\_Bajo} 20 - 80
        \item \textit{Medio\_Alto} 60 - 120
        \item \textit{Alto} 100 - 200
  \end{itemize}
\end{itemize}

Las variables de entrada son T, E, y I.
Mientras que la variable de salida es TF.  

\subsection{Segunda etapa}

En esta etapa utilizamos las variables lingüísticas:

\begin{itemize}
\item El tiempo de lectura (TF).

\item El tiempo libre (TL).

  Rango de TL: [0..14] horas.
    
  Conjuntos borrosos:
    
  \begin{itemize}
        \item \textit{Poco} 0 - 6
        \item \textit{Medio} 4 - 10
        \item \textit{Alto} 8 - 14
  \end{itemize}
        
\item El plazo límite (PM)

  Rango de PM: [1..31] días.

  Conjuntos borrosos:
  
  \begin{itemize}
        \item \textit{Bajo} 1 - 25
        \item \textit{Alto} 15 - 31
  \end{itemize}
  
\item La saturación (GS).

  Rango de GS: [0..1] $\times 10^2$ por ciento.

  Conjuntos borrosos:
  
  \begin{itemize}
        \item \textit{Bajo} 0 - 35
        \item \textit{Medio} 20 - 60
        \item \textit{Alto} 50 - 100
  \end{itemize}
\end{itemize}

Las variables de entrada son TF, TL, y PM.
Mientras que la variable de salida es GS.  


\section{Reglas}

Para la 1º etapa utilizamos las siguientes reglas:

\begin{itemize}
    \item \textbf{R1}: Si L es \textit{Grande}, E es \textit{Joven\_Adulto}, y I es \textit{Medio}, entonces TF es \textit{Medio\_Bajo}.
    \item \textbf{R2}: Si L es \textit{Grande}, E es \textit{Joven} y I es \textit{Medio}, entonces TF es \textit{Medio\_Alto}.
    \item \textbf{R3}: Si L es \textit{Grande}, E es \textit{Adulto} y I es \textit{Bajo}, entonces TF es \textit{Alto}.
    \item \textbf{R4}: Si L es \textit{Medio}, E es \textit{Joven\_Adulto} y I es \textit{Alto}, entonces TF es \textit{Medio\_Bajo}.
    \item \textbf{R5}: Si I es \textit{Bajo}, entonces TF es \textit{Alto}.
    \item \textbf{R6}: Si L es \textit{Pequeño}, entonces TF es \textit{Bajo}.
\end{itemize}

En esta etapa, utilizamos la técnica del valor medio del máximo del área.

\vspace{2 mm}

En la 2º etapa aplicamos las reglas:

\begin{itemize}
    \item \textbf{R7}: Si TF es \textit{Bajo}, y PM es \textit{Alto}, entonces GS es \textit{Medio}.
    \item \textbf{R8}: Si TF es \textit{Medio\_Alto} o \textit{Alto}, y TL es \textit{Poco}, entonces GS es \textit{Alto}.
    \item \textbf{R9}: Si TF es \textit{Medio\_Bajo}, TL es \textit{Medio}, y PM es \textit{Bajo}, entonces GS es \textit{Medio}.
    \end{itemize}

En la segunda etapa, empleamos el método de defuzzificación centroide.
\vspace{2 mm}

Al trascribir las reglas a Fispro dividimos \textbf{R8}, en dos:
\begin{itemize}
    \item \textbf{R8.1}: si TF es \textit{Alto} y TL es \textit{Poco}, GS es \textit{Alto}.
    \item \textbf{R8.2}: si TF es \textit{Medio\_Alto} y TL es \textit{Poco}, entonces GS es \textit{Alto}.
    \end{itemize}
    
Mientras que a las demás, las escribimos como las planteamos originalmente.


\section{Caso de ejemplo}

Considerando la siguiente información:

\begin{itemize}
  \item Una novela de 668 páginas, L = 668.
  \item La edad del lector de alrededor 46 años, E = 46.
  \item Grado de interés que llega sólo al 35\%, I = 35.
  \end{itemize}

Concluimos que el tiempo esperado de lectura es aproximadamente de 195 horas.
Se disparan las reglas \textbf{R1}, \textbf{R3}, y \textbf{R5}.

Luego, tomando el valor \textit{crisp} de TF anterior, TF = 195. Más los siguientes datos:

\begin{itemize}
  \item El lector tiene 4.9 horas libres por día, TL = 4.9.
  \item Un plazo límite de 16 días, PM = 16.
\end{itemize}

Obtenemos que el grado de saturación es 97.9\%. 
Vemos que solo se dispara la regla \textbf{R8.1}.

\section{Observaciones}

\begin{itemize}
  \item En la defuzzificación utilizamos como T-Norma el mínimo.
  \item Solo consideramos a los libros dentro del genero literario de cuentos y novelas.
  \item El tiempo libre que tiene el lector por día, lo utiliza para la lectura del libro en cuestión.
  \item El grado de saturación se refiere al esfuerzo diario realizado por el lector para concluir su lectura dentro del plazo límite
    que tiene estipulado. Consideramos que la lectura del libro se completa (observemos que si dejamos libre el hecho de si se concluye o no la lectura, se deberían agregar mas variables a nuestro análisis).
  \end{itemize}

\section{Integrantes}
\begin{itemize}
  \item Bonet, Javier.
  \item Catacora, Joel.
  \item Oviedo, Juan Manuel.
\end{itemize}

\end{document}
